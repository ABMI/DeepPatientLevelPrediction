% Options for packages loaded elsewhere
\PassOptionsToPackage{unicode}{hyperref}
\PassOptionsToPackage{hyphens}{url}
%
\documentclass[
]{article}
\title{Building Deep Learning Models}
\author{Jenna Reps, Egill Fridgeirsson, Chungsoo Kim, Henrik John, Seng
Chan You, Xiaoyong Pan}
\date{2022-04-03}

\usepackage{amsmath,amssymb}
\usepackage{lmodern}
\usepackage{iftex}
\ifPDFTeX
  \usepackage[T1]{fontenc}
  \usepackage[utf8]{inputenc}
  \usepackage{textcomp} % provide euro and other symbols
\else % if luatex or xetex
  \usepackage{unicode-math}
  \defaultfontfeatures{Scale=MatchLowercase}
  \defaultfontfeatures[\rmfamily]{Ligatures=TeX,Scale=1}
\fi
% Use upquote if available, for straight quotes in verbatim environments
\IfFileExists{upquote.sty}{\usepackage{upquote}}{}
\IfFileExists{microtype.sty}{% use microtype if available
  \usepackage[]{microtype}
  \UseMicrotypeSet[protrusion]{basicmath} % disable protrusion for tt fonts
}{}
\makeatletter
\@ifundefined{KOMAClassName}{% if non-KOMA class
  \IfFileExists{parskip.sty}{%
    \usepackage{parskip}
  }{% else
    \setlength{\parindent}{0pt}
    \setlength{\parskip}{6pt plus 2pt minus 1pt}}
}{% if KOMA class
  \KOMAoptions{parskip=half}}
\makeatother
\usepackage{xcolor}
\IfFileExists{xurl.sty}{\usepackage{xurl}}{} % add URL line breaks if available
\IfFileExists{bookmark.sty}{\usepackage{bookmark}}{\usepackage{hyperref}}
\hypersetup{
  pdftitle={Building Deep Learning Models},
  pdfauthor={Jenna Reps, Egill Fridgeirsson, Chungsoo Kim, Henrik John, Seng Chan You, Xiaoyong Pan},
  hidelinks,
  pdfcreator={LaTeX via pandoc}}
\urlstyle{same} % disable monospaced font for URLs
\usepackage[margin=1in]{geometry}
\usepackage{color}
\usepackage{fancyvrb}
\newcommand{\VerbBar}{|}
\newcommand{\VERB}{\Verb[commandchars=\\\{\}]}
\DefineVerbatimEnvironment{Highlighting}{Verbatim}{commandchars=\\\{\}}
% Add ',fontsize=\small' for more characters per line
\usepackage{framed}
\definecolor{shadecolor}{RGB}{248,248,248}
\newenvironment{Shaded}{\begin{snugshade}}{\end{snugshade}}
\newcommand{\AlertTok}[1]{\textcolor[rgb]{0.94,0.16,0.16}{#1}}
\newcommand{\AnnotationTok}[1]{\textcolor[rgb]{0.56,0.35,0.01}{\textbf{\textit{#1}}}}
\newcommand{\AttributeTok}[1]{\textcolor[rgb]{0.77,0.63,0.00}{#1}}
\newcommand{\BaseNTok}[1]{\textcolor[rgb]{0.00,0.00,0.81}{#1}}
\newcommand{\BuiltInTok}[1]{#1}
\newcommand{\CharTok}[1]{\textcolor[rgb]{0.31,0.60,0.02}{#1}}
\newcommand{\CommentTok}[1]{\textcolor[rgb]{0.56,0.35,0.01}{\textit{#1}}}
\newcommand{\CommentVarTok}[1]{\textcolor[rgb]{0.56,0.35,0.01}{\textbf{\textit{#1}}}}
\newcommand{\ConstantTok}[1]{\textcolor[rgb]{0.00,0.00,0.00}{#1}}
\newcommand{\ControlFlowTok}[1]{\textcolor[rgb]{0.13,0.29,0.53}{\textbf{#1}}}
\newcommand{\DataTypeTok}[1]{\textcolor[rgb]{0.13,0.29,0.53}{#1}}
\newcommand{\DecValTok}[1]{\textcolor[rgb]{0.00,0.00,0.81}{#1}}
\newcommand{\DocumentationTok}[1]{\textcolor[rgb]{0.56,0.35,0.01}{\textbf{\textit{#1}}}}
\newcommand{\ErrorTok}[1]{\textcolor[rgb]{0.64,0.00,0.00}{\textbf{#1}}}
\newcommand{\ExtensionTok}[1]{#1}
\newcommand{\FloatTok}[1]{\textcolor[rgb]{0.00,0.00,0.81}{#1}}
\newcommand{\FunctionTok}[1]{\textcolor[rgb]{0.00,0.00,0.00}{#1}}
\newcommand{\ImportTok}[1]{#1}
\newcommand{\InformationTok}[1]{\textcolor[rgb]{0.56,0.35,0.01}{\textbf{\textit{#1}}}}
\newcommand{\KeywordTok}[1]{\textcolor[rgb]{0.13,0.29,0.53}{\textbf{#1}}}
\newcommand{\NormalTok}[1]{#1}
\newcommand{\OperatorTok}[1]{\textcolor[rgb]{0.81,0.36,0.00}{\textbf{#1}}}
\newcommand{\OtherTok}[1]{\textcolor[rgb]{0.56,0.35,0.01}{#1}}
\newcommand{\PreprocessorTok}[1]{\textcolor[rgb]{0.56,0.35,0.01}{\textit{#1}}}
\newcommand{\RegionMarkerTok}[1]{#1}
\newcommand{\SpecialCharTok}[1]{\textcolor[rgb]{0.00,0.00,0.00}{#1}}
\newcommand{\SpecialStringTok}[1]{\textcolor[rgb]{0.31,0.60,0.02}{#1}}
\newcommand{\StringTok}[1]{\textcolor[rgb]{0.31,0.60,0.02}{#1}}
\newcommand{\VariableTok}[1]{\textcolor[rgb]{0.00,0.00,0.00}{#1}}
\newcommand{\VerbatimStringTok}[1]{\textcolor[rgb]{0.31,0.60,0.02}{#1}}
\newcommand{\WarningTok}[1]{\textcolor[rgb]{0.56,0.35,0.01}{\textbf{\textit{#1}}}}
\usepackage{graphicx}
\makeatletter
\def\maxwidth{\ifdim\Gin@nat@width>\linewidth\linewidth\else\Gin@nat@width\fi}
\def\maxheight{\ifdim\Gin@nat@height>\textheight\textheight\else\Gin@nat@height\fi}
\makeatother
% Scale images if necessary, so that they will not overflow the page
% margins by default, and it is still possible to overwrite the defaults
% using explicit options in \includegraphics[width, height, ...]{}
\setkeys{Gin}{width=\maxwidth,height=\maxheight,keepaspectratio}
% Set default figure placement to htbp
\makeatletter
\def\fps@figure{htbp}
\makeatother
\setlength{\emergencystretch}{3em} % prevent overfull lines
\providecommand{\tightlist}{%
  \setlength{\itemsep}{0pt}\setlength{\parskip}{0pt}}
\setcounter{secnumdepth}{5}
\usepackage{fancyhdr}
\pagestyle{fancy}
\fancyhead{}
\fancyfoot[CO,CE]{PatientLevelPrediction Package Version 5.0.2}
\fancyfoot[CO,CE]{DeepPatientLevelPrediction Package Version 0.0.1}
\fancyfoot[LE,RO]{\thepage}
\renewcommand{\headrulewidth}{0.4pt}
\renewcommand{\footrulewidth}{0.4pt}
\ifLuaTeX
  \usepackage{selnolig}  % disable illegal ligatures
\fi

\begin{document}
\maketitle

{
\setcounter{tocdepth}{3}
\tableofcontents
}
\hypertarget{introduction}{%
\section{Introduction}\label{introduction}}

Patient level prediction aims to use historic data to learn a function
between an input (a patient's features such as age/gender/comorbidities
at index) and an output (whether the patient experienced an outcome
during some time-at-risk). Deep learning is example of the the current
state-of-the-art classifiers that can be implemented to learn the
function between inputs and outputs.

Deep Learning models are widely used to automatically learn high-level
feature representations from the data, and have achieved remarkable
results in image processing, speech recognition and computational
biology. Recently, interesting results have been shown using large
observational healthcare data (e.g., electronic healthcare data or
claims data), but more extensive research is needed to assess the power
of Deep Learning in this domain.

This vignette describes how you can use the Observational Health Data
Sciences and Informatics (OHDSI)
\href{http://github.com/OHDSI/PatientLevelPrediction}{\texttt{PatientLevelPrediction}}
package and
\href{http://github.com/OHDSI/DeepPatientLevelPrediction}{\texttt{DeepPatientLevelPrediction}}
package to build Deep Learning models. This vignette assumes you have
read and are comfortable with building patient level prediction models
as described in the
\href{https://github.com/OHDSI/PatientLevelPrediction/blob/main/inst/doc/BuildingPredictiveModels.pdf}{\texttt{BuildingPredictiveModels}
vignette}. Furthermore, this vignette assumes you are familiar with Deep
Learning methods.

\hypertarget{background}{%
\section{Background}\label{background}}

Deep Learning models are build by stacking an often large number of
neural network layers that perform feature engineering steps, e.g
embedding, and are collapsed in a final softmax layer (basically a
logistic regression layer). These algorithms need a lot of data to
converge to a good representation, but currently the sizes of the large
observational healthcare databases are growing fast which would make
Deep Learning an interesting approach to test within OHDSI's
\href{https://academic.oup.com/jamia/article/25/8/969/4989437}{Patient-Level
Prediction Framework}. The current implementation allows us to perform
research at scale on the value and limitations of Deep Learning using
observational healthcare data.

In the package we have used
\href{https://cran.r-project.org/web/packages/torch/index.html}{torch}
and
\href{https://cran.r-project.org/web/packages/tabnet/index.html}{tabnet}
but we invite the community to add other backends.

Many network architectures have recently been proposed and we have
implemented a number of them, however, this list will grow in the near
future. It is important to understand that some of these architectures
require a 2D data matrix,
i.e.~\textbar patient\textbar x\textbar feature\textbar, and others use
a 3D data matrix
\textbar patient\textbar x\textbar feature\textbar x\textbar time\textbar.
The \href{www.github.com/ohdsi/FeatureExtraction}{FeatureExtraction
Package} has been extended to enable the extraction of both data formats
as will be described with examples below.

Note that training Deep Learning models is computationally intensive,
our implementation therefore supports both GPU and CPU. It will
automatically check whether there is GPU or not in your computer. A GPU
is highly recommended for Deep Learning!

\hypertarget{non-temporal-architectures}{%
\section{Non-Temporal Architectures}\label{non-temporal-architectures}}

We implemented the following non-temporal (2D data matrix)
architectures:

\begin{verbatim}
1) ...
\end{verbatim}

For the above two methods, we implemented support for a stacked
autoencoder and a variational autoencoder to reduce the feature
dimension as a first step. These autoencoders learn efficient data
encodings in an unsupervised manner by stacking multiple layers in a
neural network. Compared to the standard implementations of LR and MLP
these implementations can use the GPU power to speed up the gradient
descent approach in the back propagation to optimize the weights of the
classifier.

\#\#Example

\hypertarget{acknowledgments}{%
\section{Acknowledgments}\label{acknowledgments}}

Considerable work has been dedicated to provide the
\texttt{DeepPatientLevelPrediction} package.

\begin{Shaded}
\begin{Highlighting}[]
\FunctionTok{citation}\NormalTok{(}\StringTok{"PatientLevelPrediction"}\NormalTok{)}
\end{Highlighting}
\end{Shaded}

\begin{verbatim}
## 
## To cite PatientLevelPrediction in publications use:
## 
## Reps JM, Schuemie MJ, Suchard MA, Ryan PB, Rijnbeek P (2018).
## "Design and implementation of a standardized framework to generate
## and evaluate patient-level prediction models using observational
## healthcare data." _Journal of the American Medical Informatics
## Association_, *25*(8), 969-975. <URL:
## https://doi.org/10.1093/jamia/ocy032>.
## 
## A BibTeX entry for LaTeX users is
## 
##   @Article{,
##     author = {J. M. Reps and M. J. Schuemie and M. A. Suchard and P. B. Ryan and P. Rijnbeek},
##     title = {Design and implementation of a standardized framework to generate and evaluate patient-level prediction models using observational healthcare data},
##     journal = {Journal of the American Medical Informatics Association},
##     volume = {25},
##     number = {8},
##     pages = {969-975},
##     year = {2018},
##     url = {https://doi.org/10.1093/jamia/ocy032},
##   }
\end{verbatim}

\textbf{Please reference this paper if you use the PLP Package in your
work:}

\href{http://dx.doi.org/10.1093/jamia/ocy032}{Reps JM, Schuemie MJ,
Suchard MA, Ryan PB, Rijnbeek PR. Design and implementation of a
standardized framework to generate and evaluate patient-level prediction
models using observational healthcare data. J Am Med Inform Assoc.
2018;25(8):969-975.}

\end{document}
